Table \ref{table:FE_RQ1} presents the unit Fixed Effect (FE) linear-log model, with period mean aggregated data, for the six health dimensions with a one-period lag of ODA. Across all health dimensions, the coefficients of the lagged ODA are negative. Specifically, a one percent increase in previously disbursed ODA, on average, reduces reproductive fatalities and teen pregnancy (RFTP) by 0.016, burden of infection and diseases (BID) by 0.008, environmental death by 0.018, malnutrition by 0.003 and burden of mental problems (BMP) by 0.016 standard deviations, respectively. Surprisingly, the coefficient of ODA is negative for health system capacity, with effect estimates of -0.011 standard deviations. It's noteworthy that although the ODA coefficients exhibit a mix of directions, ODA is only weakly significant for RFTP.


Table \ref{table:DPM_RQ1} shows the results of the linear-log Fixed Effect Cross-Lag Panel Model (FE-CLPM), based on the maximum likelihood estimation of the structural equation model (ML-SEM). The model incorporates the autoregressive parameters (impact of previous health outcome on current health outcome) and fewer control variables, excluding population and population density. In this, model and similar to the fixed effect model above, the coefficients of the one-period lagged ODA exhibit the anticipated direction of effects (negative) for all health dimensions, except healthcare capacity and responsiveness. Specifically, a one-period lagged ODA, on average, reduces RFTP by 0.041, BID by 0.030, malnutrition by 0.042, and environmental death by 0.035 standard deviations respectively. Notably, and unlike the fixed effect model, the impact of ODA is highly significant for RFTP at 1\% and mildly significant for BID, malnutrition at 5\%, and environmental death at 10\% p\_value. Table \ref{table:DPM_RQ1_Log} shows the log-log (elasticity) version of FE-CLDM, the result is stable across all health dimensions, with some of the mentioned health dimensions becoming even more significant. It is noteworthy that the coefficients of all autoregressive parameters are highly significant, indicating the strong influence of the initial health situation on contemporaneous health and the appropriateness of the dynamic model approach for this study. 


The result of the local projection method is consistent with previously presented methods, showing the anticipated direction of effect for all health dimensions, as shown in Figure \ref{fig:Local_projection}. Accordingly, contemporaneous ODA has both intermediate and long-term effects on the burden of infection and diseases (BID), environmental death, and reproductive fatalities (RFTP), though some effects are weaker than others. Specifically, ODA exhibits the strongest intermediate and long-term effects on reproductive fatalities, with the effect becoming statistically significant after four years and lasting until the seventh year. While ODA is weakly significant for environmental death after six years, the significance level increases in the seventh and eighth years. Similarly, the impact of ODA on BID is weakly significant, starting from the fourth year and continuing until the seventh year. Notably, the impact of ODA changes direction on all health dimensions except mental burden, after the seventh year.    


Research Question two: 
As shown in Table \ref{table:FE_RQ2} for unit fixed effect, the impact of ODA varies both across health dimensions and regions. Accordingly, while ODA shows the expected direction of effect on all health dimensions for SSA region except for mental problem (BMP) and health system capacity (HSCR), only infection and disease (BID) is weakly significant for the region. Notably, the impact of ODA is mildly significant at 5\% for reproductive fatality (RFTP) in the Middle East and North Africa (MENA) region. Moreover, ODA has weak significance (at 10\%) on mental problems and environmental death (at 5\%) p\_value in Europe respectively. Aside from these notable differences, the impact of ODA is not significantly different across regions on the rest of the health dimensions.

Unlike the unit FE model, FE-CLDM presented in Figure \ref{table:DPM_RQ2} shows a slightly different result. While the direction of effect is similar to FE model, ODA now has a weak significant effect on environmental death in SSA region. Moreover and surprisingly, the coefficients of ODA across all health dimensions are significant for Latin America and the Caribbean (LAC), albeit positive estimates for environmental death and negative for health system capacity.


Research question three: 
This section presents the result, in Table \ref{tab:MediationModel}, on the mediating role of social protection in the impact of ODA on health. For modeling convenience and in line with the causal order: \(ODA_{it-4} \rightarrow SP_{it-3} \rightarrow HO_{it}\), discussed in the methodology section, year-wave data is employed in the analysis. All relevant variables were demeaned within to reflect the unit FE method, and respective lag and log transformations of relevant variables were performed before modeling. Additionally, a time trend variable was included like other models and the standard errors are corrected using robust error and the bootstrap procedure with 5000 replications across all health dimensions.

The mediation result, in Table \ref{tab:MediationModel}, shows that ODA has a highly significant impact on reproductive fatalities, with an average reduction of 0.008 standard deviations in response to a 1\% change in ODA. While ODA exhibits the desired direction of coefficients for other health dimensions, except mental health, they are statistically insignificant. Since the direct effect of all ODA on health remains the same as the total effect, and all confidence intervals of the indirect effect include 0, the analysis does not provide sufficient evidence that social protection has a significant mediating role in the effect of ODA on all health dimensions. It is important to note that the causal assumption used to derive the indirect effect assumes that treatment (ODA allocation) precedes the mediator (Social Protection), and the mediator, in turn, precedes the outcome (health). Any deviation from these assumptions, such as variations in lag periods, the direction of the relationship, or high missingness of social protection data, could imply potential underestimation or overestimation of the indirect effect. Therefore, this analysis serves as a simple policy guide for the optimal allocation of ODA to enhance health, recognizing its limitations.


Note: Total ODA tend to have more effect on reproductive fatality while social infrastructure ODA is more effective for infection and diseases.


Robustness Checks:
To assess the robustness of the models presented above, the thesis estimated four models with two approaches: unit FE with time trend as control and FE-CLDM using social infrastructure ODA, as opposed to total ODA in the main models for both the main research question and the regional variation question. All robustness models employ a similar set of covariates as the main models. Similar to the main models, these robustness models were implemented with aggregated data, and the results are presented in Appendix Table \ref{table:DPM_Robst_RQ1}, \ref{table:UnitFE_Robst_Region_RQ2}, \ref{table:DPM_Robst_Region_RQ2} and \ref{table:UnitFE_Robst_Region_RQ2}.
Accordingly, on the impact of ODA on health outcome, the unit FE model in Appendix Table \ref{table:UnitFE_Robst_Region_RQ2} shows social infrastructure ODA is only significant for infection and diseases (BID) at 5\% p\_value. Moreover, while reproductive fatality and environmental death have the desired direction of effect (negative), they are insignificant. Surprisingly, the FE-CLPM approach on the Table \ref{table:DPM_Robst_RQ1} shows strong significance of social infrastructure ODA for all health dimensions, except the burden of mental problem (BMP). While the latter result is consistent with the main models, HSCR and malnutrition are now strongly significant. 

Regarding the regional heterogeneity of the social infrastructure ODA, the FE model, on Table \ref{table:UnitFE_Robst_Region_RQ2} shows a weakly significant effect on environmental death for SSA region, and infection and disease for LAC region. The FE-CLDM model, on the other hand in Table \ref{table:DPM_Robst_Region_RQ2}, shows more complex result. Accordingly, social infrastructure ODA is found on reproductive fatality, infection and disease and HSCR for East Asia and Pacific (EAP), SSA, and MENA regions, all significant at 5\% p\_value. Notably, social infrastructure ODA impact on was also significant on malnutrition for EAP and SSA regions. A possible explanation for these sporadic significance results is that the social infrastructure ODA data employed is commitment, as opposed to disbursement data for total ODA in the main model. 


The key findings from the analyses are summarized as follows:

\begin{enumerate}[i]
    \item The impact of Official Development Assistance (ODA) on health exhibits variations based on the type of ODA, econometric model employed, and health dimensions considered. Despite these variations, both total and social infrastructure ODA demonstrate significant impacts on specific health dimensions, such as reproductive fatality, infections and diseases, and environmental death health dimensions. These consistent findings are observed across the fixed effect, FE-CLPM, local projection, and robustness models. Therefore, against the null hypothesis, there is evidence that ODA has an impact on at least one of the health dimensions being considered. Moreover, social infrastructure ODA tends to be more impacting on infection and diseases, compared to total ODA which is more effective on reproductive fatality.
    Surprisingly, all coefficients of ODA for health system capacity (HSCR) are negative across models.

    \item Utilizing year-wave data, the local projection models presented in Figure \ref{fig:Local_projection} reveal that ODA has both intermediate (4 years) and long-term (six years) effects on reproductive fatalities and infections and diseases, with the effect being stronger in the formal than the later. The impact of ODA on environmental death is long-term, with the effect materializing six years after ODA disbursement. 

    \item Regading the regional heterogeneity of both total and social infrastructure ODA impact, there are variations in the pattern of effect across models. Regardless, there is evidence that ODA impact is weakly significantly different on infection and disease for SSA compared to non-SSA regions, as shown in all models. Moreover, social infrastructure ODA tends to be most impactful on reproductive fatality in SSA, EAP, and MENA regions. This type of ODA is only significant on HSCR and positive for the EAP region. Therefore, there is weak evidence that ODA impact varies among regions. 
    
   % In terms of regional variation in effect, the impact of total ODA significantly differs for Sub-Saharan Africa (SSA) countries, especially in the context of infections and diseases. Conversely, social infrastructure ODA demonstrates greater significance for non-SSA countries, particularly across infections and diseases, reproductive fatalities, and environmental death.

    \item The analysis of the mediating role of social protection does not provide sufficient evidence to support the notion that social protection plays an indirect or mediating role in the impact of ODA on health. Due to the limitations associated with social protection data and causal order assumptions, there exists a potential for underestimation or overestimation. Therefore, these findings are presented as a pragmatic policy guide for optimizing ODA allocation rather than definitive conclusions.
\end{enumerate}